\documentclass[11pt,a4paper]{article}
\usepackage{amsmath,amssymb,amsthm}
\usepackage{geometry}
\usepackage{hyperref}
\usepackage{booktabs}
\usepackage{enumitem}
\usepackage{xcolor}
\geometry{margin=1in}
\setlength{\parindent}{0pt}
\setlength{\parskip}{8pt}

\title{\textbf{Notes}}
\author{}
\date{\today}

\begin{document}

\maketitle
\tableofcontents 
\newpage

\section{State Space Models and Filtering Methods}

\subsection{Overview and Big Picture}
Given the most general time series, \\
\textbf{Evolution (Process Model):}
\begin{equation}
\mathbf{x}_t = f(\mathbf{x}_{t-1}) + \mathbf{w}_t, \quad \mathbf{w}_t \sim p_w(\cdot)
\end{equation}

\textbf{Measurement (Observation Model):}
\begin{equation}
\mathbf{y}_t = h(\mathbf{x}_t) + \mathbf{v}_t, \quad \mathbf{v}_t \sim p_v(\cdot)
\end{equation}
Filtering methods estimate hidden states of dynamical systems from noisy measurements. The choice of filter depends on three key properties: \textbf{linearity of dynamics}, \textbf{process noise distribution}, and \textbf{measurement noise distribution}.

\textbf{Key Insight:} Process noise $\mathbf{w}_t$ and measurement noise $\mathbf{v}_t$ can have \textbf{different distributions} independently.

\begin{center}
\begin{tabular}{@{}llll@{}}
\toprule
\textbf{Linearity} & \textbf{Process Noise} & \textbf{Measurement Noise} & \textbf{Filter} \\
\midrule
Linear & Gaussian & Gaussian & Kalman (exact, optimal) \\
Nonlinear & Gaussian & Gaussian & EKF / UKF (approximate) \\
Any & Non-Gaussian & Gaussian & Particle / Hybrid \\
Any & Gaussian & Non-Gaussian & Particle / Robust Kalman \\
Any & Non-Gaussian & Non-Gaussian & Particle / Flow \\
\bottomrule
\end{tabular}
\end{center}

\textbf{Filtering Hierarchy:}
\begin{itemize}
\item \textbf{Exact (Linear + Both Gaussian):} Kalman Filter
\item \textbf{Approximate (Nonlinear + Both Gaussian):}
\begin{itemize}
\item EKF: Linearization via Jacobian (1st-order)
\item UKF: Sigma point transform (3rd-order)
\end{itemize}
\item \textbf{Sampling-based (At Least One Non-Gaussian):}
\begin{itemize}
\item Standard Particle Filter: Stochastic sampling + resampling
\item Particle Flow Filters: Deterministic ODE evolution
\item Hybrid Methods: Rao-Blackwellized PF, Robust Kalman, Gaussian Sum
\end{itemize}
\end{itemize}

All methods solve the Bayesian filtering problem: compute $p(\mathbf{x}_t | \mathbf{y}_{1:t})$ recursively. where $p(x_t|y_{1:t})$ is the filtering distribution or posterior distribution of the hidden state at time t given all observations up to time t.

\subsection{General Framework}

State space models (SSM) describe dynamical systems with hidden states observed through noisy measurements.

\textbf{Evolution (Process Model):}
\begin{equation}
\mathbf{x}_t = f(\mathbf{x}_{t-1}) + \mathbf{w}_t, \quad \mathbf{w}_t \sim p_w(\cdot)
\end{equation}

\textbf{Measurement (Observation Model):}
\begin{equation}
\mathbf{y}_t = h(\mathbf{x}_t) + \mathbf{v}_t, \quad \mathbf{v}_t \sim p_v(\cdot)
\end{equation}

where $f(\cdot)$ is the evolution function, $h(\cdot)$ is the measurement function, $p_w(\cdot)$ is the \textbf{process noise distribution}, and $p_v(\cdot)$ is the \textbf{measurement noise distribution}. These two noise distributions are \textbf{independent} and can be of different types (e.g., one Gaussian, one non-Gaussian).

\textbf{Goal:} Estimate hidden state $\mathbf{x}_t$ given measurements $\mathbf{y}_{1:t}$ by computing the posterior $p(\mathbf{x}_t | \mathbf{y}_{1:t})$.

\subsection{\color{red} To add, discussion for noiseless case}

\subsection{Kalman Filter: Linear-Gaussian Case}

For linear dynamics with \textbf{both} process and measurement noise Gaussian:
\begin{align}
\mathbf{x}_t &= \hat{\mathbf{F}} \mathbf{x}_{t-1} + \mathbf{w}_t, \quad \mathbf{w}_t \sim \mathcal{N}(0, \mathbf{Q}) \quad \text{(process noise)} \\
\mathbf{y}_t &= \hat{\mathbf{H}} \mathbf{x}_t + \mathbf{v}_t, \quad \mathbf{v}_t \sim \mathcal{N}(0, \mathbf{R}) \quad \text{(measurement noise)}
\end{align}

where $\hat{\mathbf{F}}$ is the state transition matrix (evolution operator), $\hat{\mathbf{H}}$ is the measurement matrix, $\mathbf{Q}$ is the process noise covariance, and $\mathbf{R}$ is the measurement noise covariance.

The Kalman filter provides the \textbf{exact, closed-form} optimal solution when all conditions hold (linear + both Gaussian).

\textbf{Notation:}
\begin{itemize}
\item $\hat{\mathbf{x}}_t^-$ (also $\hat{\mathbf{x}}_{t|t-1}$, $\bar{\mathbf{x}}_t$, or $\mathbf{x}_t^f$) = \textit{a priori} state estimate (mean of the predicted distribution)
\item $\hat{\mathbf{x}}_t^+$ (also $\hat{\mathbf{x}}_{t|t}$, $\hat{\mathbf{x}}_t$, or $\mathbf{x}_t^a$) = \textit{a posteriori} state estimate (mean of the predicted distribution)
\item $\mathbf{P}_t^-$ (also $\mathbf{P}_{t|t-1}$ or $\bar{\mathbf{P}}_t$) = \textit{a priori} error covariance
\item $\mathbf{P}_t^+$ (also $\mathbf{P}_{t|t}$ or $\mathbf{P}_t$) = \textit{a posteriori} error covariance
\end{itemize}

\textbf{Prediction Step:}
\begin{align}
\hat{\mathbf{x}}_t^- &= \mathbf{\hat F} \hat{\mathbf{x}}_{t-1}^+ \\
\mathbf{P}_t^- &= \mathbf{\hat F} \mathbf{P}_{t-1}^+ \mathbf{\hat F}^T + \mathbf{Q}
\end{align}

\textbf{Correction Step:}
\begin{align}
\mathbf{\hat K}_t &= \mathbf{P}_t^- \mathbf{H}^T \left[\mathbf{H} \mathbf{P}_t^- \mathbf{H}^T + \mathbf{R}\right]^{-1} \quad \text{(Kalman gain)} \\
\hat{\mathbf{x}}_t^+ &= \hat{\mathbf{x}}_t^- + \mathbf{K}_t (\mathbf{y}_t - \mathbf{H}\hat{\mathbf{x}}_t^-) \\
\mathbf{\hat P}_t^+ &= (\mathbf{I} - \mathbf{K}_t \mathbf{H}) \mathbf{P}_t^-
\end{align}

where $\mathbf{K}_t$ is the Kalman gain (optimal adaptive learning rate matrix) and $(\mathbf{y}_t - \mathbf{H}\hat{\mathbf{x}}_t^-)$ is the innovation (also called residual or measurement surprise).

The Kalman gain balances model prediction versus measurement based on their respective uncertainties: larger $\mathbf{R}$ (noisy sensors) reduces gain; larger $\mathbf{P}_t^-$ (uncertain prediction) increases gain.


Consider a linear-Gaussian state-space model at time \( t \):

\begin{itemize}
    \item State vector: \( \mathbf{x}_t \in \mathbb{R}^n \), the hidden state.
    \item Observation vector: \( \mathbf{z}_t \in \mathbb{R}^m \), the noisy measurement.
    \item Transition matrix: \( F \in \mathbb{R}^{n \times n} \), linear dynamics operator.
    \item Observation matrix: \( H \in \mathbb{R}^{m \times n} \), linear measurement operator.
    \item Process noise: \( \mathbf{v}_t \sim \mathcal{N}(\mathbf{0}, Q) \), with covariance \( Q \in \mathbb{R}^{n \times n} \).
    \item Measurement noise: \( \mathbf{w}_t \sim \mathcal{N}(\mathbf{0}, R) \), with covariance \( R \in \mathbb{R}^{m \times m} \).
\end{itemize}

The model equations are:
\begin{align}
    \mathbf{x}_t &= F \mathbf{x}_{t-1} + \mathbf{v}_t, \label{eq:state_transition} \\
    \mathbf{z}_t &= H \mathbf{x}_t + \mathbf{w}_t. \label{eq:observation}
\end{align}

% Prior distribution
The predicted state (prior) given observations up to \( t-1 \) is:
\begin{itemize}
    \item Mean: \( \hat{\mathbf{x}}_{t|t-1} = F \hat{\mathbf{x}}_{t-1|t-1} \),
    \item Covariance: \( P_{t|t-1} = F P_{t-1|t-1} F^\top + Q \).
\end{itemize}
Thus, the prior distribution is:
\[
p(\mathbf{x}_t \mid \mathbf{z}_{1:t-1}) = \mathcal{N}(\hat{\mathbf{x}}_{t|t-1}, P_{t|t-1}).
\]
The likelihood is:
\[
p(\mathbf{z}_t \mid \mathbf{x}_t) = \mathcal{N}(H \mathbf{x}_t, R).
\]

% Joint Gaussian distribution
The posterior \( p(\mathbf{x}_t \mid \mathbf{z}_{1:t}) \propto p(\mathbf{z}_t \mid \mathbf{x}_t) p(\mathbf{x}_t \mid \mathbf{z}_{1:t-1}) \) is Gaussian. To derive the Kalman gain, consider the joint Gaussian distribution of \( \mathbf{x}_t \) and \( \mathbf{z}_t \) (conditional on \( \mathbf{z}_{1:t-1} \)):
\[
\begin{bmatrix}
\mathbf{x}_t \\
\mathbf{z}_t
\end{bmatrix}
\sim \mathcal{N} \left(
\begin{bmatrix}
\hat{\mathbf{x}}_{t|t-1} \\
H \hat{\mathbf{x}}_{t|t-1}
\end{bmatrix},
\begin{bmatrix}
P_{t|t-1} & P_{t|t-1} H^\top \\
H P_{t|t-1} & H P_{t|t-1} H^\top + R
\end{bmatrix}
\right).
\]

% Conditional Gaussian formula
For jointly Gaussian vectors \( \mathbf{x} \) and \( \mathbf{z} \) with means \( \boldsymbol{\mu}_x, \boldsymbol{\mu}_z \), covariances \( \Sigma_{xx}, \Sigma_{zz} \), and cross-covariance \( \Sigma_{xz} \), the conditional distribution is:
\[
\mathbf{x} \mid \mathbf{z} \sim \mathcal{N} \left( \boldsymbol{\mu}_x + \Sigma_{xz} \Sigma_{zz}^{-1} (\mathbf{z} - \boldsymbol{\mu}_z), \ \Sigma_{xx} - \Sigma_{xz} \Sigma_{zz}^{-1} \Sigma_{zx} \right).
\]

% Applying to Kalman filter
Substitute:
\begin{itemize}
    \item \( \boldsymbol{\mu}_x = \hat{\mathbf{x}}_{t|t-1} \),
    \item \( \boldsymbol{\mu}_z = H \hat{\mathbf{x}}_{t|t-1} \),
    \item \( \Sigma_{xx} = P_{t|t-1} \),
    \item \( \Sigma_{zz} = H P_{t|t-1} H^\top + R \),
    \item \( \Sigma_{xz} = P_{t|t-1} H^\top \), \( \Sigma_{zx} = H P_{t|t-1} \).
\end{itemize}

The posterior mean is:
\begin{align}
\hat{\mathbf{x}}_{t|t} &= \hat{\mathbf{x}}_{t|t-1} + P_{t|t-1} H^\top (H P_{t|t-1} H^\top + R)^{-1} (\mathbf{z}_t - H \hat{\mathbf{x}}_{t|t-1}). \label{eq:posterior_mean}
\end{align}

% Defining the Kalman gain
Define the Kalman gain as:
\[
K_t = P_{t|t-1} H^\top (H P_{t|t-1} H^\top + R)^{-1}. \label{eq:kalman_gain}
\]

The innovation (residual) is \( \tilde{\mathbf{y}}_t = \mathbf{z}_t - H \hat{\mathbf{x}}_{t|t-1} \), so the update becomes:
\[
\hat{\mathbf{x}}_{t|t} = \hat{\mathbf{x}}_{t|t-1} + K_t \tilde{\mathbf{y}}_t. \label{eq:state_update}
\]

% Posterior covariance
The posterior covariance is:
\begin{align}
P_{t|t} &= P_{t|t-1} - P_{t|t-1} H^\top (H P_{t|t-1} H^\top + R)^{-1} H P_{t|t-1} \notag \\
&= (I - K_t H) P_{t|t-1}, \label{eq:posterior_cov}
\end{align}
where \( I \) is the identity matrix. This Joseph stabilized form ensures numerical stability.

% Intuition
\section*{Intuition}
The Kalman gain \( K_t \) balances the trust between the prior prediction \( \hat{\mathbf{x}}_{t|t-1} \) (with uncertainty \( P_{t|t-1} \)) and the measurement \( \mathbf{z}_t \) (with noise \( R \)). A large \( R \) (noisy measurement) reduces \( K_t \), relying more on the prior; a large \( P_{t|t-1} \) (uncertain prediction) increases \( K_t \), favoring the measurement. This minimizes the trace of \( P_{t|t} \), ensuring the minimum mean-squared error (MMSE) estimate.


\subsection{Extended Kalman Filter (EKF)}

\textbf{When:} Nonlinear $f, h$ with \textbf{both} process and measurement noise Gaussian.

\textbf{Strategy:} Linearize via first-order Taylor expansion around current estimate.

\textbf{Jacobians:}
\begin{equation}
\mathbf{F}_t = \left.\frac{\partial f}{\partial \mathbf{x}}\right|_{\mathbf{x} = \hat{\mathbf{x}}_{t-1}^+}, \quad
\mathbf{H}_t = \left.\frac{\partial h}{\partial \mathbf{x}}\right|_{\mathbf{x} = \hat{\mathbf{x}}_t^-}
\end{equation}

Apply Kalman filter equations using time-varying $\mathbf{F}_t, \mathbf{H}_t$.

\textbf{Prediction:}
\begin{align}
\hat{\mathbf{x}}_t^- &= f(\hat{\mathbf{x}}_{t-1}^+) \\
\mathbf{P}_t^- &= \mathbf{F}_t \mathbf{P}_{t-1}^+ \mathbf{F}_t^T + \mathbf{Q}
\end{align}

\textbf{Correction:} Same as Kalman filter but with $\mathbf{H}_t$ and predicted measurement $\hat{\mathbf{y}}_t^- = h(\hat{\mathbf{x}}_t^-)$.

\textbf{Limitations:} Only first-order accurate; can diverge for strong nonlinearity; requires Jacobian computation (may be expensive or analytically intractable); assumes Gaussian noises.

\subsection{Unscented Kalman Filter (UKF)}

\textbf{When:} Nonlinear $f, h$ with \textbf{both} noises Gaussian; better for strong nonlinearity than EKF.

\textbf{Strategy:} Propagate uncertainty through nonlinear functions using \textbf{sigma points} (also called unscented points) - deterministically chosen samples that capture mean and covariance.

\textbf{Sigma Points:} For state with mean $\hat{\mathbf{x}}$ and covariance $\mathbf{P}$, generate $2n+1$ points:
\begin{align}
\mathcal{X}^{(0)} &= \hat{\mathbf{x}} \\
\mathcal{X}^{(i)} &= \hat{\mathbf{x}} + \left(\sqrt{(n+\lambda)\mathbf{P}}\right)_i, \quad i=1,\ldots,n \\
\mathcal{X}^{(i)} &= \hat{\mathbf{x}} - \left(\sqrt{(n+\lambda)\mathbf{P}}\right)_{i-n}, \quad i=n+1,\ldots,2n
\end{align}

where $\mathcal{X}^{(i)}$ denotes the $i$-th sigma point, $n$ is the state dimension, and $\lambda$ is a scaling parameter.

\textbf{Transform:} Pass each sigma point through actual nonlinear function:
\begin{equation}
\mathcal{Y}^{(i)} = h(\mathcal{X}^{(i)})
\end{equation}

\textbf{Reconstruct:} Compute weighted mean and covariance of transformed points:
\begin{align}
\hat{\mathbf{y}} &= \sum_{i=0}^{2n} W^{(i)} \mathcal{Y}^{(i)} \\
\mathbf{P}_y &= \sum_{i=0}^{2n} W^{(i)} (\mathcal{Y}^{(i)} - \hat{\mathbf{y}})(\mathcal{Y}^{(i)} - \hat{\mathbf{y}})^T
\end{align}

where $W^{(i)}$ are predetermined weights.

\textbf{Advantages:} No Jacobian needed (derivative-free); captures up to 3rd-order accuracy (vs. EKF's 1st-order); same $O(n^3)$ complexity as EKF.

\textbf{Limitations:} Still assumes both noises are Gaussian.

\subsection{Particle Filter (Sequential Monte Carlo)}

\textbf{When:} Arbitrary nonlinearity and/or \textbf{at least one} non-Gaussian noise; multimodal distributions.

\textbf{Strategy:} Represent posterior distribution using $N$ weighted particles (samples):
\begin{equation}
p(\mathbf{x}_t | \mathbf{y}_{1:t}) \approx \sum_{i=1}^{N} w_t^{(i)} \delta(\mathbf{x}_t - \mathbf{x}_t^{(i)})
\end{equation}

where $\mathbf{x}_t^{(i)}$ is the $i$-th particle (sample state), $w_t^{(i)}$ is its weight, and $\delta(\cdot)$ is the Dirac delta function.

\textbf{Algorithm (SIS: Sequential Importance Sampling with Resampling):}

1. \textbf{Prediction (stochastic):} For each particle $i = 1, \ldots, N$:
\begin{equation}
\mathbf{x}_t^{(i)} = f(\mathbf{x}_{t-1}^{(i)}) + \mathbf{w}_t^{(i)}, \quad \mathbf{w}_t^{(i)} \sim p_w
\end{equation}
Particles evolve by random sampling from the process noise distribution $p_w$ (can be non-Gaussian).

2. \textbf{Update Weights:} Compute likelihood and normalize:
\begin{equation}
w_t^{(i)} \propto w_{t-1}^{(i)} \cdot p(\mathbf{y}_t | \mathbf{x}_t^{(i)}), \quad \sum_i w_t^{(i)} = 1
\end{equation}

where $p(\mathbf{y}_t | \mathbf{x}_t^{(i)})$ is the measurement likelihood (depends on $p_v$, can be non-Gaussian).

3. \textbf{Resampling (stochastic):} Resample particles with replacement according to weights to avoid degeneracy (also called particle depletion or weight collapse). High-weight particles get duplicated; low-weight particles die out.

4. \textbf{Estimate:} Compute weighted average:
\begin{equation}
\hat{\mathbf{x}}_t = \sum_{i=1}^N w_t^{(i)} \mathbf{x}_t^{(i)}
\end{equation}

\textbf{Advantages:} Handles arbitrary nonlinearity; handles non-Gaussian noise in \textbf{either or both} process and measurement; can represent multimodal distributions; asymptotically exact as $N \to \infty$.

\textbf{Limitations:} Computationally expensive ($O(Nn^2)$); curse of dimensionality for high-dimensional states ($n > 10$); particle degeneracy issues despite resampling.

\subsection{Particle Flow Filters (Advanced)}

\textbf{When:} Similar to particle filters but with \textbf{deterministic} particle evolution to avoid degeneracy.

\textbf{Key Difference:} Standard particle filters use \textbf{stochastic} sampling and resampling (Monte Carlo). Particle flow filters move particles \textbf{deterministically} via continuous-time flow.

\textbf{Strategy:} Move particles via ODE from prior to posterior, rather than random sampling.

\textbf{Homotopy Flow:} Define parameter $\lambda \in [0,1]$ where $\lambda=0$ is prior $p(\mathbf{x}_t | \mathbf{y}_{1:t-1})$ and $\lambda=1$ is posterior $p(\mathbf{x}_t | \mathbf{y}_{1:t})$.

Particles evolve via:
\begin{equation}
\frac{d\mathbf{x}^{(i)}}{d\lambda} = \mathbf{u}(\mathbf{x}^{(i)}, \lambda)
\end{equation}

where $\mathbf{u}$ is a drift function designed to morph the prior into the posterior.

\textbf{Example (Log-Homotopy):} 
\begin{equation}
p_\lambda(\mathbf{x}) \propto p(\mathbf{x}_t | \mathbf{y}_{1:t-1}) \cdot p(\mathbf{y}_t | \mathbf{x})^\lambda
\end{equation}

As $\lambda$ increases from 0 to 1, likelihood is gradually incorporated.

\textbf{Common Methods:}
\begin{itemize}
\item Daum-Huang Filter: Original exact ODE formulation
\item Feedback Particle Filter: Control-theoretic approach
\item Stein Variational Gradient Descent (SVGD): ML/variational inference perspective
\end{itemize}

\textbf{Advantages:} No particle degeneracy; no resampling needed; better for high-dimensional problems; smoother convergence; connects to optimal transport theory; handles non-Gaussian noises.

\textbf{Limitations:} Higher computational cost ($O(N^2n^2)$ per step); requires gradients $\nabla_\mathbf{x} \log p(\mathbf{y}_t | \mathbf{x})$; more complex implementation.

\subsection{Hybrid and Special Methods}

When noise types differ or have special structure, specialized methods can be more efficient than full particle filtering.

\textbf{Rao-Blackwellized (Marginalized) Particle Filter:}
If state decomposes as $\mathbf{x}_t = [\mathbf{x}_t^{(1)}, \mathbf{x}_t^{(2)}]^T$ where part $(2)$ is conditionally linear-Gaussian given part $(1)$:
\begin{itemize}
\item Use particle filter for $\mathbf{x}_t^{(1)}$ (non-Gaussian part)
\item Use Kalman filter for $\mathbf{x}_t^{(2)} | \mathbf{x}_t^{(1)}$ (Gaussian part)
\item Much more efficient than full particle filtering
\end{itemize}

\textbf{Robust Kalman Filter:}
For mostly Gaussian measurement noise with occasional outliers:
\begin{itemize}
\item Run standard Kalman filter
\item Detect outliers via chi-squared test on innovation
\item Reject or downweight suspicious measurements
\end{itemize}

\textbf{Gaussian Sum Filter:}
For noise that is a mixture of Gaussians: $p(\mathbf{w}) = \sum_{k=1}^K \alpha_k \mathcal{N}(\mathbf{w}; \boldsymbol{\mu}_k, \mathbf{\Sigma}_k)$
\begin{itemize}
\item Run $K$ parallel Kalman filters (one per mixture component)
\item Combine outputs via weighted sum
\end{itemize}

\subsection{Summary and Selection Guide}

\begin{center}
\begin{tabular}{@{}lllllll@{}}
\toprule
\textbf{Filter} & $f, h$ & \textbf{Proc. Noise} & \textbf{Meas. Noise} & \textbf{Method} & \textbf{Complexity} & \textbf{Accuracy} \\
\midrule
Kalman & Linear & Gaussian & Gaussian & Exact & $O(n^3)$ & Optimal \\
EKF & Nonlinear & Gaussian & Gaussian & 1st-order lin. & $O(n^3)$ & Good (mild) \\
UKF & Nonlinear & Gaussian & Gaussian & Sigma points & $O(n^3)$ & Better (strong) \\
Particle & Arbitrary & Arbitrary & Arbitrary & Stochastic MC & $O(Nn^2)$ & Asymptotic \\
Flow & Arbitrary & Arbitrary & Arbitrary & Deterministic ODE & $O(N^2n^2)$ & Asymptotic \\
Hybrid & Mixed & Mixed & Mixed & Combined & Varies & Problem-specific \\
\bottomrule
\end{tabular}
\end{center}

\textbf{Decision Tree:}
\begin{itemize}
\item Linear $f, h$ + both Gaussian $\rightarrow$ \textbf{Kalman Filter} (exact, optimal)
\item Nonlinear $f$ or $h$ + both Gaussian:
\begin{itemize}
\item Mild nonlinearity $\rightarrow$ \textbf{EKF} (fast, needs Jacobian)
\item Strong nonlinearity $\rightarrow$ \textbf{UKF} (better accuracy, derivative-free)
\end{itemize}
\item At least one non-Gaussian noise:
\begin{itemize}
\item Standard case or low-D ($n < 10$) $\rightarrow$ \textbf{Particle Filter}
\item Special structure (part linear-Gaussian) $\rightarrow$ \textbf{Rao-Blackwellized PF}
\item Occasional outliers only $\rightarrow$ \textbf{Robust Kalman}
\item Mixture of Gaussians $\rightarrow$ \textbf{Gaussian Sum Filter}
\item High-D or degeneracy issues $\rightarrow$ \textbf{Particle Flow}
\end{itemize}
\end{itemize}

\textbf{Core Insight:} All methods solve the same Bayesian filtering problem $p(\mathbf{x}_t | \mathbf{y}_{1:t})$ using different computational strategies. The Kalman gain concept (optimal weighting of prediction vs. measurement based on uncertainty) underlies all approaches, appearing explicitly in Kalman/EKF/UKF and implicitly in particle methods. Process and measurement noises are independent and can have different distributions.
\newpage
\section{Theoretical intuition}
In this section, I want to lay out clear theoretical foundation for the problem at hand. Let us start with the most general case for a time series, \\
\textbf{Evolution (Process Model):}
\begin{equation}
\mathbf{x}_t = f(\mathbf{x}_{t-1}) + \mathbf{w}_t, \quad \mathbf{w}_t \sim p_w(\cdot)
\end{equation}

\textbf{Measurement (Observation Model):}
\begin{equation}
\mathbf{y}_t = h(\mathbf{x}_t) + \mathbf{v}_t, \quad \mathbf{v}_t \sim p_v(\cdot)
\end{equation}

where $f(\cdot)$ is the evolution function, $h(\cdot)$ is the measurement function, $p_w(\cdot)$ is the \textbf{process noise distribution}, and $p_v(\cdot)$ is the \textbf{measurement noise distribution}. These two noise distributions are \textbf{independent} and can be of different types (e.g., one Gaussian, one non-Gaussian). 

\subsection{One-dimensional problems}
Suppose the observation is perfect, we ask given the distribution of $\mathbf{x}_{t-1}$, $\rho_{t-1}(x)$, how do we compute the distribution at $t$, namely $\rho_t$. Let us use one dimension as a demonstration,
\begin{equation}
\begin{split}
\rho_t(x_t)=\int dw_t\;\; p_w(w_t)\int dx_{t-1} \; \;  \rho_t(x_{t-1}) \; \delta\Big(x_t-  f(x_{t-1}) + w_t\Big)
\end{split}
\end{equation}
This is very difficult to compute for arbitrary function $f(\cdot)$ in high dimension, even without observation noise, due to the difficulty of convergence in high dimension.  If somehow we are luck enough to deal with a system where we can expand $f(x_{t-1})$. 
\begin{equation}
\begin{split}
\rho_t(x_t)=&\int dw_t\;\; p_w(w_t)\int dx_{t-1} \; \;  \rho_t(x_{t-1}) \; \delta\Big(x_t-  f(x_{t-1}) + w_t\Big) \\
=& \int dw_t\;\; p_w(w_t)\int dx_{t-1} \; \;  \rho_t(x_{t-1}) \; \delta\Big(x_t- (1+ f_t) x_{t-1}+ w_t\Big) \\
=&  \int dx_{t-1} \; \;  \rho_t(x_{t-1}) p_w((1+ f_t) x_{t-1}-x_t)
\end{split}
\end{equation}
where $f_{t-1}=\partial_{x}f(x)|_{x_{t-1}}$. We can make the problem even easier, such that $f$ becomes a constant. The problem is straightforward but difficult. 
However if we have imperfect observation, things become much more involved. 
\begin{equation}
x_t =f x_{t-1} +  w_t, \quad w_t \sim p_w(\cdot)
\end{equation}
\begin{equation}
y_t = hx_t + v_t, \quad v_t \sim p_v(\cdot)
\end{equation}
now the goal, is to, given the information about ${y_1,y_2.....,y_t}$, determine the probability $P(x_t|y_{1:t})$. The core idea of filtering method is,
\begin{equation}
\begin{split}
p(x_t | y_{1:t-1}) &= \int dx_{t-1} \;\;p(x_t | x_{t-1}) p(x_{t-1} | y_{1:t-1}) \\
p(x_t | y_{1:t}) &= p(y_t | x_t) · p(x_t | y_{1:t-1})
\end{split}
\end{equation}
Now, this looks like not much more complicated than the case of perfect observation, namely the same number of integrals for certain discretization scheme.\\
\textbf{Derivation of 1D Kalman Filter}


\textbf{State Evolution:}
\begin{equation}
x_t = a x_{t-1} + w_t, \quad w_t \sim \mathcal{N}(0, Q)
\end{equation}

\textbf{Observation:}
\begin{equation}
y_t = h x_t + v_t, \quad v_t \sim \mathcal{N}(0, R)
\end{equation}

\textbf{Initial Condition:}
\begin{equation}
p(x_0) = \mathcal{N}(x_0 | m_0, P_0)
\end{equation}

\textbf{Goal}
Compute the filtering distribution $p(x_t | y_{1:t}) = \mathcal{N}(x_t | m_t, P_t)$ recursively.

\textbf{Key Fact}
The filtering distribution remains Gaussian at all times. We only need to track the mean $m_t$ and variance $P_t$.

\textbf{Predict Step}

\textbf{Goal:} Compute $p(x_t | y_{1:t-1})$ from $p(x_{t-1} | y_{1:t-1})$.

Given $p(x_{t-1} | y_{1:t-1}) = \mathcal{N}(x_{t-1} | m_{t-1}, P_{t-1})$, we have:
\begin{equation}
p(x_t | y_{1:t-1}) = \int dx_{t-1} \, p(x_t | x_{t-1}) \, p(x_{t-1} | y_{1:t-1})
\end{equation}

From the state evolution model:
\begin{equation}
p(x_t | x_{t-1}) = \mathcal{N}(x_t | a x_{t-1}, Q)
\end{equation}

\textbf{Computing the predicted mean:}
\begin{align}
m_t^- &= \mathbb{E}[x_t | y_{1:t-1}] \\
&= \mathbb{E}[\mathbb{E}[x_t | x_{t-1}, y_{1:t-1}] | y_{1:t-1}] \\
&= \mathbb{E}[a x_{t-1} | y_{1:t-1}] \\
&= a m_{t-1}
\end{align}

\textbf{Computing the predicted variance:}
\begin{align}
P_t^- &= \text{Var}[x_t | y_{1:t-1}] \\
&= \mathbb{E}[\text{Var}[x_t | x_{t-1}, y_{1:t-1}] | y_{1:t-1}] + \text{Var}[\mathbb{E}[x_t | x_{t-1}, y_{1:t-1}] | y_{1:t-1}] \\
&= \mathbb{E}[Q | y_{1:t-1}] + \text{Var}[a x_{t-1} | y_{1:t-1}] \\
&= Q + a^2 P_{t-1}
\end{align}

\textbf{Result:}
\begin{equation}
p(x_t | y_{1:t-1}) = \mathcal{N}(x_t | m_t^-, P_t^-)
\end{equation}
where
\begin{equation}
\boxed{m_t^- = a m_{t-1}, \quad P_t^- = a^2 P_{t-1} + Q}
\end{equation}

\textbf{Update Step}

\textbf{Goal:} Incorporate observation $y_t$ using Bayes' rule.

\begin{equation}
p(x_t | y_{1:t}) = \frac{p(y_t | x_t) \, p(x_t | y_{1:t-1})}{p(y_t | y_{1:t-1})}
\end{equation}

From the observation model:
\begin{equation}
p(y_t | x_t) = \mathcal{N}(y_t | h x_t, R)
\end{equation}

We need to compute the product of two Gaussians:
\begin{align}
p(x_t | y_{1:t}) &\propto \mathcal{N}(y_t | h x_t, R) \cdot \mathcal{N}(x_t | m_t^-, P_t^-) \\
&\propto \exp\left(-\frac{1}{2}\frac{(y_t - h x_t)^2}{R}\right) \exp\left(-\frac{1}{2}\frac{(x_t - m_t^-)^2}{P_t^-}\right)
\end{align}

Expanding the exponents:
\begin{align}
-\frac{1}{2}\left[\frac{(y_t - h x_t)^2}{R} + \frac{(x_t - m_t^-)^2}{P_t^-}\right] &= -\frac{1}{2}\left[\frac{y_t^2 - 2 y_t h x_t + h^2 x_t^2}{R} + \frac{x_t^2 - 2 x_t m_t^- + (m_t^-)^2}{P_t^-}\right]
\end{align}

Collecting terms in $x_t^2$ and $x_t$:
\begin{equation}
= -\frac{1}{2}\left[\left(\frac{h^2}{R} + \frac{1}{P_t^-}\right) x_t^2 - 2\left(\frac{h y_t}{R} + \frac{m_t^-}{P_t^-}\right) x_t + \text{const}\right]
\end{equation}

This is a Gaussian in $x_t$. Completing the square:

\textbf{Posterior variance:}
\begin{align}
\frac{1}{P_t} &= \frac{h^2}{R} + \frac{1}{P_t^-} \\
P_t &= \frac{1}{\frac{h^2}{R} + \frac{1}{P_t^-}} = \frac{R P_t^-}{h^2 P_t^- + R}
\end{align}

Define the \textbf{Kalman gain:}
\begin{equation}
K_t = \frac{P_t^- h}{h^2 P_t^- + R}
\end{equation}

Then:
\begin{equation}
P_t = (1 - K_t h) P_t^-
\end{equation}

\textbf{Posterior mean:}
\begin{align}
\frac{m_t}{P_t} &= \frac{h y_t}{R} + \frac{m_t^-}{P_t^-} \\
m_t &= P_t \left(\frac{h y_t}{R} + \frac{m_t^-}{P_t^-}\right)
\end{align}

Substituting $P_t = (1 - K_t h) P_t^-$:
\begin{align}
m_t &= (1 - K_t h) P_t^- \left(\frac{h y_t}{R} + \frac{m_t^-}{P_t^-}\right) \\
&= (1 - K_t h) \left(\frac{P_t^- h y_t}{R} + m_t^-\right) \\
&= (1 - K_t h) m_t^- + (1 - K_t h) \frac{P_t^- h y_t}{R}
\end{align}

Using $K_t = \frac{P_t^- h}{h^2 P_t^- + R}$, we have $\frac{P_t^- h}{R} = K_t \frac{h^2 P_t^- + R}{R} = K_t (1 + \frac{h^2 P_t^-}{R})$.

After algebra:
\begin{equation}
m_t = m_t^- + K_t (y_t - h m_t^-)
\end{equation}

\textbf{Result:}
\begin{equation}
\boxed{
\begin{aligned}
K_t &= \frac{P_t^- h}{h^2 P_t^- + R} \\
m_t &= m_t^- + K_t (y_t - h m_t^-) \\
P_t &= (1 - K_t h) P_t^-
\end{aligned}
}
\end{equation}

\textbf{Summary: 1D Kalman Filter Algorithm}

\textbf{Initialize:} $m_0, P_0$

\textbf{For} $t = 1, 2, 3, \ldots$:

\textbf{Predict:}
\begin{align}
m_t^- &= a m_{t-1} \\
P_t^- &= a^2 P_{t-1} + Q
\end{align}

\textbf{Update:}
\begin{align}
K_t &= \frac{P_t^- h}{h^2 P_t^- + R} \\
m_t &= m_t^- + K_t (y_t - h m_t^-) \\
P_t &= (1 - K_t h) P_t^-
\end{align}

\textbf{Output:} $p(x_t | y_{1:t}) = \mathcal{N}(x_t | m_t, P_t)$

From here it is trivial to generalize to higher dimension.


\textbf{Stochastic Volatility Model (1D)}

\textbf{State Evolution:}
\begin{equation}
x_t = \alpha x_{t-1} + \sigma w_t, \quad w_t \sim \mathcal{N}(0, 1)
\end{equation}

\textbf{Observation:}
\begin{equation}
y_t = \beta \exp\left(\frac{x_t}{2}\right) v_t, \quad v_t \sim \mathcal{N}(0, 1)
\end{equation}

\textbf{Parameters:}
\begin{itemize}
\item $x_t$ = log-volatility (hidden state)
\item $y_t$ = observed returns
\item $\alpha \in (0, 1)$ = persistence parameter
\item $\sigma$ = volatility of volatility
\item $\beta$ = scale parameter
\end{itemize}

\textbf{Key Features:}
\begin{itemize}
\item \textbf{Nonlinear observation:} $\exp(x_t/2)$ makes observation nonlinear in state
\item \textbf{Non-Gaussian likelihood:} $p(y_t | x_t)$ is not Gaussian in $x_t$
\item Linear state evolution + nonlinear/non-Gaussian observations $\Rightarrow$ Kalman Filter fails
\end{itemize}


\textbf{Range-Bearing Observation Model}

A nonlinear tracking problem with polar observations.

\textbf{State (Cartesian coordinates):}
\begin{equation}
\mathbf{x}_t = \begin{bmatrix} p_x \\ p_y \\ v_x \\ v_y \end{bmatrix}_t
\end{equation}
where $(p_x, p_y)$ = position, $(v_x, v_y)$ = velocity.

\textbf{State Evolution (linear):}
\begin{equation}
\mathbf{x}_t = \mathbf{F} \mathbf{x}_{t-1} + \mathbf{w}_t, \quad \mathbf{w}_t \sim \mathcal{N}(\mathbf{0}, \mathbf{Q})
\end{equation}
\begin{equation}
\mathbf{F} = \begin{bmatrix}
1 & 0 & \Delta t & 0 \\
0 & 1 & 0 & \Delta t \\
0 & 0 & 1 & 0 \\
0 & 0 & 0 & 1
\end{bmatrix}
\end{equation}

\textbf{Observation (nonlinear):}
\begin{equation}
\mathbf{y}_t = \begin{bmatrix} r_t \\ \theta_t \end{bmatrix} = h(\mathbf{x}_t) + \mathbf{v}_t, \quad \mathbf{v}_t \sim \mathcal{N}(\mathbf{0}, \mathbf{R})
\end{equation}
where:
\begin{align}
r_t &= \sqrt{p_x^2 + p_y^2} \quad \text{(range)} \\
\theta_t &= \arctan(p_y, p_x) \quad \text{(bearing)}
\end{align}

\textbf{Key Features:}
\begin{itemize}
\item Highly nonlinear observation function (Cartesian $\to$ polar transformation)
\item Linear state dynamics but nonlinear measurement
\item Standard benchmark for nonlinear filtering methods
\end{itemize}


\textbf{Model Reminder:}
\begin{align}
x_t &= \alpha x_{t-1} + \sigma w_t, \quad w_t \sim \mathcal{N}(0, 1) \\
y_t &= \beta \exp\left(\frac{x_t}{2}\right) v_t, \quad v_t \sim \mathcal{N}(0, 1)
\end{align}

\subsubsection*{1. Extended Kalman Filter (EKF)}

\textbf{Idea:} Linearize the nonlinear observation function around the current state estimate.

\textbf{State Transition:}
\begin{equation}
f(x_{t-1}) = \alpha x_{t-1}, \quad F_t = \frac{\partial f}{\partial x}\Big|_{x=m_{t-1}} = \alpha
\end{equation}

\textbf{Observation Function:}
\begin{equation}
h(x_t) = \beta \exp\left(\frac{x_t}{2}\right) \cdot 0 = 0 \quad \text{(mean of observation given } x_t\text{)}
\end{equation}

Wait, this needs clarification. The observation model is:
\begin{equation}
y_t | x_t \sim \mathcal{N}\left(0, \beta^2 \exp(x_t)\right)
\end{equation}

So $h(x_t) = 0$ (mean) but variance depends on $x_t$.

For EKF, we linearize the observation model. Since $y_t = \beta \exp(x_t/2) v_t$, we have:
\begin{equation}
p(y_t | x_t) = \mathcal{N}(y_t | 0, \beta^2 \exp(x_t))
\end{equation}

The log-likelihood is:
\begin{equation}
\log p(y_t | x_t) = -\frac{1}{2}\log(2\pi) - \frac{1}{2}\log(\beta^2 \exp(x_t)) - \frac{y_t^2}{2\beta^2 \exp(x_t)}
\end{equation}
\begin{equation}
= -\frac{1}{2}\log(2\pi) - \frac{1}{2}\log(\beta^2) - \frac{x_t}{2} - \frac{y_t^2}{2\beta^2 \exp(x_t)}
\end{equation}

For EKF, we linearize around the predicted mean $m_t^-$:
\begin{equation}
y_t \approx h(m_t^-) + H_t (x_t - m_t^-) + \text{noise}
\end{equation}

where $h(x) = 0$ and 
\begin{equation}
H_t = \frac{\partial h}{\partial x}\Big|_{x=m_t^-} = 0
\end{equation}

\textbf{Problem:} The standard EKF linearization doesn't work well here because the observation mean is always 0. 

\textbf{Alternative EKF Approach:} Linearize the observation variance. We approximate:
\begin{equation}
\sigma_y^2(x_t) = \beta^2 \exp(x_t) \approx \beta^2 \exp(m_t^-) \cdot \exp(x_t - m_t^-)
\end{equation}

Using first-order Taylor expansion: $\exp(x_t - m_t^-) \approx 1 + (x_t - m_t^-)$

This gives:
\begin{equation}
\sigma_y^2(x_t) \approx \beta^2 \exp(m_t^-) [1 + (x_t - m_t^-)]
\end{equation}

\textbf{EKF Algorithm:}

\textbf{Predict:}
\begin{align}
m_t^- &= \alpha m_{t-1} \\
P_t^- &= \alpha^2 P_{t-1} + \sigma^2
\end{align}

\textbf{Update:}

The EKF update for this model is non-standard because the observation noise variance depends on the state. A common approximation is to use the predicted state to evaluate the observation noise:
\begin{equation}
R_t = \beta^2 \exp(m_t^-)
\end{equation}

Then apply a modified update based on the innovation:
\begin{equation}
\nu_t = y_t^2 - \beta^2 \exp(m_t^-) \quad \text{(innovation in squared observation)}
\end{equation}

The linearization of $\mathbb{E}[y_t^2 | x_t] = \beta^2 \exp(x_t)$ gives:
\begin{equation}
H_t = \beta^2 \exp(m_t^-)
\end{equation}

\begin{equation}
S_t = H_t P_t^- H_t + \text{Var}[y_t^2] \approx 2\beta^4 \exp(2m_t^-)
\end{equation}

\begin{align}
K_t &= P_t^- H_t S_t^{-1} \\
m_t &= m_t^- + K_t \nu_t \\
P_t &= (1 - K_t H_t) P_t^-
\end{align}

\textbf{Note:} EKF for stochastic volatility is notoriously poor because the linearization is inadequate for the exponential nonlinearity.

\subsubsection*{2. Unscented Kalman Filter (UKF)}

\textbf{Idea:} Use sigma points to propagate mean and covariance through the nonlinearity without linearization.

\textbf{Predict Step:}

\textbf{1. Generate sigma points from } $p(x_{t-1} | y_{1:t-1}) = \mathcal{N}(m_{t-1}, P_{t-1})$:

For 1D, use 3 sigma points:
\begin{align}
\chi_{t-1}^{(0)} &= m_{t-1} \\
\chi_{t-1}^{(1)} &= m_{t-1} + \sqrt{(1 + \lambda) P_{t-1}} \\
\chi_{t-1}^{(2)} &= m_{t-1} - \sqrt{(1 + \lambda) P_{t-1}}
\end{align}

with weights:
\begin{align}
W_0^{(m)} &= \frac{\lambda}{1 + \lambda}, \quad W_0^{(c)} = \frac{\lambda}{1 + \lambda} + (1 - \alpha^2 + \beta) \\
W_i^{(m)} &= W_i^{(c)} = \frac{1}{2(1 + \lambda)}, \quad i = 1, 2
\end{align}

where $\lambda = \alpha^2(1 + \kappa) - 1$ (typical: $\alpha = 10^{-3}, \kappa = 0, \beta = 2$).

\textbf{2. Propagate through state dynamics:}
\begin{equation}
\chi_{t|t-1}^{(i)} = \alpha \chi_{t-1}^{(i)}, \quad i = 0, 1, 2
\end{equation}

\textbf{3. Compute predicted mean and covariance:}
\begin{align}
m_t^- &= \sum_{i=0}^{2} W_i^{(m)} \chi_{t|t-1}^{(i)} \\
P_t^- &= \sum_{i=0}^{2} W_i^{(c)} (\chi_{t|t-1}^{(i)} - m_t^-)^2 + \sigma^2
\end{align}

\textbf{Update Step:}

\textbf{1. Generate sigma points from predicted distribution:}
\begin{align}
\chi_t^{(0)} &= m_t^- \\
\chi_t^{(1)} &= m_t^- + \sqrt{(1 + \lambda) P_t^-} \\
\chi_t^{(2)} &= m_t^- - \sqrt{(1 + \lambda) P_t^-}
\end{align}

\textbf{2. Propagate through observation function:}

For stochastic volatility, the observation mean is 0, but we work with the variance. A common approach is to use the transformed observation $z_t = y_t^2$:
\begin{equation}
\gamma_t^{(i)} = \beta^2 \exp(\chi_t^{(i)}), \quad i = 0, 1, 2
\end{equation}

\textbf{3. Compute predicted observation statistics:}
\begin{align}
\hat{z}_t &= \sum_{i=0}^{2} W_i^{(m)} \gamma_t^{(i)} \\
S_t &= \sum_{i=0}^{2} W_i^{(c)} (\gamma_t^{(i)} - \hat{z}_t)^2 + 2\beta^4 \exp(2m_t^-) \\
C_t &= \sum_{i=0}^{2} W_i^{(c)} (\chi_t^{(i)} - m_t^-)(\gamma_t^{(i)} - \hat{z}_t)
\end{align}

\textbf{4. Kalman update:}
\begin{align}
K_t &= C_t S_t^{-1} \\
m_t &= m_t^- + K_t (y_t^2 - \hat{z}_t) \\
P_t &= P_t^- - K_t S_t K_t^T
\end{align}

\textbf{Note:} UKF better captures the nonlinearity than EKF but still assumes the posterior remains approximately Gaussian.

\subsubsection*{3. Particle Filter}

\textbf{Idea:} Represent the filtering distribution $p(x_t | y_{1:t})$ using weighted particles (samples).

\textbf{Representation:}
\begin{equation}
p(x_t | y_{1:t}) \approx \sum_{i=1}^{N} w_t^{(i)} \delta(x_t - x_t^{(i)})
\end{equation}

where $\{x_t^{(i)}, w_t^{(i)}\}_{i=1}^{N}$ are particles and weights, with $\sum_{i=1}^{N} w_t^{(i)} = 1$.

\textbf{Sequential Importance Sampling (SIS):}

\textbf{Initialization} ($t = 0$):
\begin{align}
x_0^{(i)} &\sim p(x_0), \quad i = 1, \ldots, N \\
w_0^{(i)} &= \frac{1}{N}
\end{align}

\textbf{For } $t = 1, 2, \ldots$:

\textbf{1. Predict (Propagate particles):}

Sample from the state transition:
\begin{equation}
x_t^{(i)} = \alpha x_{t-1}^{(i)} + \sigma \epsilon^{(i)}, \quad \epsilon^{(i)} \sim \mathcal{N}(0, 1)
\end{equation}

\textbf{2. Update (Weight particles):}

Compute importance weights based on the observation likelihood:
\begin{equation}
p(y_t | x_t^{(i)}) = \mathcal{N}(y_t | 0, \beta^2 \exp(x_t^{(i)}))
\end{equation}
\begin{equation}
= \frac{1}{\sqrt{2\pi \beta^2 \exp(x_t^{(i)})}} \exp\left(-\frac{y_t^2}{2\beta^2 \exp(x_t^{(i)})}\right)
\end{equation}

Update weights:
\begin{equation}
\tilde{w}_t^{(i)} = w_{t-1}^{(i)} \cdot p(y_t | x_t^{(i)})
\end{equation}

\textbf{3. Normalize weights:}
\begin{equation}
w_t^{(i)} = \frac{\tilde{w}_t^{(i)}}{\sum_{j=1}^{N} \tilde{w}_t^{(j)}}
\end{equation}

\textbf{4. Resampling (when needed):}

Check effective sample size:
\begin{equation}
N_{\text{eff}} = \frac{1}{\sum_{i=1}^{N} (w_t^{(i)})^2}
\end{equation}

If $N_{\text{eff}} < N_{\text{threshold}}$ (e.g., $N/2$), resample:
\begin{itemize}
\item Draw $N$ new particles $\{\bar{x}_t^{(i)}\}_{i=1}^{N}$ from $\{x_t^{(i)}\}_{i=1}^{N}$ with probabilities $\{w_t^{(i)}\}_{i=1}^{N}$
\item Set $x_t^{(i)} = \bar{x}_t^{(i)}$ and $w_t^{(i)} = 1/N$ for all $i$
\end{itemize}

\textbf{State Estimates:}

Filtered mean:
\begin{equation}
\hat{x}_t = \sum_{i=1}^{N} w_t^{(i)} x_t^{(i)}
\end{equation}

Filtered variance:
\begin{equation}
\hat{P}_t = \sum_{i=1}^{N} w_t^{(i)} (x_t^{(i)} - \hat{x}_t)^2
\end{equation}

\textbf{Key Issues:}
\begin{itemize}
\item \textbf{Particle degeneracy:} After several iterations, most weights become negligible
\item \textbf{Sample impoverishment:} After resampling, many particles are duplicates
\item \textbf{Curse of dimensionality:} Number of particles needed grows exponentially with state dimension
\end{itemize}

\textbf{Advantages:}
\begin{itemize}
\item No linearity or Gaussianity assumptions
\item Asymptotically exact as $N \to \infty$
\item Can handle arbitrary nonlinearities
\end{itemize}

\section{Literature review for part one}

\subsection{}

 

\end{document}